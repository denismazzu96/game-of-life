\documentclass{article}
% \usepackage{graphicx}
% \usepackage{enumitem}
% \usepackage{amssymb}
% \usepackage{float}
% \usepackage{tabularx}
% \usepackage[htt]{hyphenat}
% \usepackage[dvipsnames]{xcolor}

\begin{document}

\title{MPI Assignment - Game of Life}
\author{Denis Mazzucato\\Vrije Universiteit Amsterdam}
\date{2019/20}

\maketitle
\vfill
\tableofcontents
\newpage


% What was the problem to solve and which algorithm(s) did you use?
% In case you changed the sequential program, what were the changes?
% What is the theoretical estimation of the performance of the parallel program? Note that the assignments might need different techniques to estimate the expected parallel performance.
% What is the experimental set-up? Always specify a complete experimental set-up, explain how you decided what measurements to take and why, and what is the relevance of each experiment or set of experiments.
% What was the procedure of taking measurements? Explain what you have measured, why, and how.
% What were the results of your measurements? Include both numbers and graphs. Execution times and speed-ups must be included in your report; however, do include (in separate tables and/or graphs) any additional relevant measurements that you've taken and analyzed.
% What do the numbers say about program performance? Explain all the results you include in the report; any unexplained graph or table is basically useless for the reader, and therefore will not be considered for grading your report.
% How do the results compare to the theoretical estimation? Focus on explaining the (possible) gaps: why they appear and how can they be alleviated.
% Is there any slow-down? Why? Is there any superlinear speedup? Why? Is there a flattening of the speed-up curve? Why?
% How does the performance scale in case there were more processors?
% What are the conclusions of your work? What is your experience with the particular language/programming model that the assignment targets? Any generic difficulties you have encountered? Any guidelines you've extracted from your work?


\section{Game of Life}

The problem to solve is a simple board game regards life evolving during a time
simulation.
The basic algorithm simply iterate each timestep and for each cell calculate
the new value based on the neighbors,
without any kind of trick or exploit.
I've used the same algorithm also in both parallel and optimized version.
By the way, some proper devices are introduced to make the most of
\textit{Das5} infrastructure.
This problem is a perfect example of what is called a stencil operation and
this performs very well with a parallel algorithm.

\subsection{Parallel algorithm}
This version of the parallel algorithm simply distributes the data over nodes.
So, in particular it distributes the grid (representing the board game) using
row-wise partitioning.
Each processor holds a slice of this grid and it needs to exchange two lines
every timestep

\subsection{Optimized algorithm}


\section{Theoretical speedup and performance}

\section{Measurement set-up}

\section{}

\end{document}